In this section we would like to introduce the timetable datasets we were working with and provide the analysis of their properties. The main reason for this analysis is that it gives some insight into the characteristics of the timetables and so may contribute to develop an oracle based method with better qualities.

\subsection{Data}

	We have obtained timetable datasets from numerous sources, in varying formats and of different types. Some of them were freely available on the Internet while others were provided by companies upon demand. Let us provide their brief description.
	
	The dataset \textit{air01} contains schedules of \textbf{domestic flights in United States} for the January of 2008. It is not comprehensive in the sense that it contains entries only for flights of some of the major airports in US. However it is large enough for our purposes (almost 300 airports). This dataset is just a fraction of the data that are freely available at the pages of American Statistical Association~\footnote{\url{http://stat-computing.org/dataexpo/2009/the-data.html}} in CSV format.
	
	Timetable \textit{cpsk} represent the \textbf{regional bus} schedules from the areas of \textbf{Ružomberok and Žilina, Slovakia}. The data were provided by the company in charge of the \textit{cp.sk} portal - Inprop s.r.o. . The timetable contains about 1900 bus stops and came in a JDF 1.9 format~\footnote{Jednotný dátový formát (JDF).}. Apart from the actual schedules, the data in JDF contain numerous other information which were not relevant for our purposes. From both timetables we have extracted subsets with a time range of one day.
	
	The \textit{gb-coach} and \textit{gb-train} timetables are freely available from National Public Transport Data Repository (NPTDR)~\footnote{\url{http://data.gov.uk}} in an ATCO-CIF format. These are not actually timetables but rather weekly snapshots of national public transport journeys made by \textbf{coach and train in Great Britain} (during certain week in year 2011). The datasets contain about 2500 stations each.
	
	The \textit{montr} dataset is part of a public feed for \textbf{Greater Montreal public transportation}, available at Google Transit Feeds~\footnote{\url{http://code.google.com/p/googletransitdatafeed/wiki/PublicFeeds}}. The data are in a GTFS format (defines relations between CSV files listing stations, routes, stop-times...) and were made available by Montreal's Agence métropolitaine de transport. Our timetable \textit{montr} corresponds to daily schedules of the Chambly-Richelieu-Carignan bus services (more than 200 bus stops).
	
	Also in GTFS format come the data of \textbf{French railways} operated by company SNCF, publicly available at their website~\footnote{\url{http://test.data-sncf.com/index.php/ter.html}}. The schedules are weekly and there were two of them: one for intercity trains and one for TER trains (regional trains). Thus the three timetables \textit{sncf-inter} (366 stations), \textit{sncf-ter} (2637 stations) and their union \textit{sncf} (2646 stations).
	
	Finally, one more country-wide railway timetable was provided by ŽSR, the company in charge of the \textbf{Slovak national railways}. This timetable was exported in a MERITS format and its time range is for one year. The number of stations in \textit{zsr} dataset is 233. \\
	
	\noindent With the help of Python and Bash scripts, we converted each of these datasets to our timetable format (described in appendix~\ref{app:formats}). This timetables were then loaded by our application TTBlazer, which can further generate sub-timetables (with less stations or smaller time range), underlying graphs and TE and TD graphs.
	
	For a summary of the used timetables' descriptions, see table~\ref{table:tt_desc} and for their main properties, refer to table~\ref{table:tt_mainpr}. \\
	
	\begin{table}[h!]
		\centering
		\footnotesize
		\begin{tabular}{c|c|c|c|c}
		%legend
			\rowcolor{tablehead}
			\textbf{Name} & \textbf{Description} & \textbf{Format} & \textbf{Provided by} & \textbf{Publicly available} \\
		%data
			\hline
			\textit{air01} & domestic flights (US) & CSV & American Stat. Assoc. & \tick \\
			\textit{cpsk} & regional bus (Ružomberok \& Žilina, SVK) & JDF 1.9 & Inprop s.r.o. & \cross \\
			\textit{gb-coach} & country-wide buses (GB) & ATCO-CIF & NPTDR & \tick \\
			\textit{gb-train} & country-wide rails (GB) & ATCO-CIF & NPTDR & \tick \\
			\textit{montr} & public transport (Montreal, CA) & GTFS & Montreal AMT & \tick \\
			\textit{sncf} & country-wide rails (FRA) & GTFS & SNCF & \tick \\
			\textit{zsr} & country-wide rails (SVK) & MERITS & ŽSR & \cross \\
		\end{tabular}
		\caption{\label{table:tt_desc} Datasets descriptions.}
		\normalsize
	\end{table}		
	
	\begin{table}[h!]
		\centering
		\small
		\begin{tabular}{c|c|c|c|c|c}
		%legend
			\rowcolor{tablehead}
			\textbf{Name} & \textbf{El. conns.} & \textbf{Cities} & \textbf{UG arcs} & \textbf{Time range} & \textbf{Height} \\
		%data
			\hline
			\textit{air01} & 601489 & 287 & 4668 & 1 month & 24374 \\
			\textit{cpsk} & 97916 & 1905 & 5093 & 1 day & 370 \\
			\textit{gb-coach} & 260710 & 2448 & 5793 & 1 week & 3140 \\
			\textit{gb-train} & 1714535 & 2555 & 8335 & 1 week & 7978 \\
			\textit{montr} & 7153 & 217 & 349  & 1 day & 363 \\
			\textit{sncf} & 416302 & 2646 & 7994 & 1 week & 2679 \\
			\textit{sncf-inter} & 22750 & 366 & 901 & 1 week & 1052 \\
			\textit{sncf-ter} & 393587 & 2637 & 7647 & 1 week & 2646 \\
			\textit{zsr} & 932052 & 233 & 588 & 1 year & 60308 \\
		\end{tabular}
		\caption{\label{table:tt_mainpr} Main properties of the timetables. The value of time range is approximate.}
		\normalsize
	\end{table}	
	
	\noindent To see better the differences in the properties of different timetable types (train, flight, bus...), we made sub-timetables with 200 cities and with the upper bound on time range being 1 day and 6 hours~\footnote{We took all elementary connections that were within our time range. From this timetable, we made an UG and its (random) sub-graph of 200 cities. Finally we selected only those elementary connections, that were on top of this sub-graph to form a timetable with 200 cities and the desired (maximal) time range.} ($thigh_{T} <$ 1 day and 6 hours) from each of our dataset. We name these datasets by appending to the original name ``\textit{-200d}''~\footnote{Similarly, throughout this thesis, suffix ``\textit{-d}'' would mean ``with daily time range'', ``\textit{-w}'' ``weekly time range'' and suffix ``\textit{-\#}'' would mean sub-timetable with \# stations.}. See table~\ref{table:tt_200mainpr} for details. \\
	
	\begin{table}[h!]
		\centering
		\small
		\begin{tabular}{c|c|c|c|c}
		%legend
			\rowcolor{tablehead}
			\textbf{Name} & \textbf{El. conns.} & \textbf{Cities} & \textbf{UG arcs} & \textbf{Height} \\
		%data
			\hline
			\textit{air01-200d} & 19010 & 200 & 3973 & 772 \\
			\textit{cpsk-200d} & 14747 & 200 & 592 & 370 \\
			\textit{gb-coach-200d} & 2760 & 200 & 564 & 498 \\
			\textit{gb-train-200d} & 24323 & 200 & 792 & 957 \\
			\textit{montr-200d} & 6841 & 200 & 320 & 355 \\
			\textit{sncf-200d} & 4192 & 200 & 611 & 269 \\		
			\textit{sncf-inter-200d} & 2172 & 200 & 493 & 128 \\
			\textit{sncf-ter-200d} & 8469 & 200 & 600 & 419 \\
			\textit{zsr-200d} & 2031 & 200 & 454 & 133 \\
		\end{tabular}
		\caption{\label{table:tt_200mainpr} 200-station sub-timetables with the time range of one day.}
		\normalsize
	\end{table}
	
	\noindent Also, to further justify our choice of using TD graphs instead of TE graphs in this thesis, we provide their space consumption comparison in table~\ref{table:tdte}. \\
	
	\begin{table}[h!]
		\centering
		\small
		\begin{tabular}{c|c|c|c|c|c|c}
		%legend
			\rowcolor{tablehead}
			 & \multicolumn{3}{>{\columncolor{tablehead}}c|}{\textbf{TD graph}} & \multicolumn{3}{>{\columncolor{tablehead}}c}{\textbf{TE graph}} \\ \hline
			\rowcolor{tablehead}
			\textbf{Name} & \textbf{Nodes} & \textbf{Arcs} & \textbf{Size (MB)} & \textbf{Nodes} & \textbf{Arcs} & \textbf{Size (MB)} \\
		%data
			\hline
			\textit{air01} & 287 & 4668 & 27 & 715211 & 1307432 & 72 \\
			\textit{cpsk} & 1905 & 5093 & 5 & 95601 & 189205 & 11 \\
			\textit{gb-coach} & 2448 & 5793 & 12 & 259589 & 512862 & 32 \\
			\textit{gb-train} & 2555 & 8335 & 79 & 2042316 & 3745751 & 263 \\
			\textit{montr} & 217 & 349 & 0.4 & 7182 & 13992 & 0.9 \\
			\textit{sncf} & 2646 & 7994 & 19 & 758867 & 1166646 & 85 \\
			\textit{sncf-inter} & 366 & 901 & 1.1 & 39765 & 60602 & 4.6 \\
			\textit{sncf-ter} & 2637 & 7647 & 18 & 720651 & 1107301 & 81 \\
			\textit{zsr} & 233 & 588 & 42 & 1706077 & 2637896 & 173 \\
		\end{tabular}
		\caption{\label{table:tdte} Space consumption of time-dependent vs. time-expanded model. The number of nodes and arcs for TD graph is the same as for the corresponding underlying graph.}
		\normalsize
	\end{table}	
	
\subsection{Analysis of properties}

	\noindent First we will take a look at the optimal connection \textit{sizes} (size is the number of elementary connections) in the timetables. For a given timetable $T$, we will denote the average optimal connection size as $\bm{\gamma_{T}}$ and will call it the \textbf{optimal connection diameter} (OC diameter). We computed an approximate OC diameter for each of our datasets by measuring an average connection size of sufficiently many OCs. The results in table~\ref{table:tt_ocdiam} indicate that the average OC size generally falls under $\sqrt{n}$.
	
	Next we would like to get an idea of the sparsity of the underlying graphs. We see from the table~\ref{table:tt_mainpr} that the graphs are pretty sparse (with the exception of \textit{air01}), but we would like to make sure that the sparsity is uniform. More specifically, we will be interested in the $\delta$-density:
	
	\begin{definition}
		\textbf{$\bm{\delta}$-density} \\
		A graph $G$ of $n$ vertices and $m$ arcs is $\delta$-dense $\iff$ $\forall G' \subseteq G, n' \geq \sqrt[4]{n}: \frac{\displaystyle m'}{\displaystyle n'} \leq \delta$
		\begin{itemize}
			\item For a timetable $T$, we will denote its \textbf{density} parameter~\footnote{Note that this has nothing to do with the frequency of elementary connections, only with the density of the underlying graph.} as $\bm{\delta_{T}} = min \{\delta|\; ug_{T}$ is $\delta$-dense$\}$
		\end{itemize}
	\end{definition}
	
	\noindent To find out at least approximate $\delta_{T}$ values for our timetables, we have randomly sampled their UGs for (connected) sub-graphs of various sizes (starting from $\sqrt[4]{n}$~\footnote{The choice of $\sqrt[4]{n}$ will be justified later, during the analysis of the algorithms.}). In table~\ref{table:ug_density} you can see the maximal density found during the sampling. \\
	
	\begin{figure}[htb]
	\centering
	\makebox[0pt][c]{
    \begin{minipage}{0.45\textwidth} 
		\centering
		\begin{tabular}{c|c|c|c}
		%legend
			\rowcolor{tablehead}
			\textbf{Name} & \textbf{$\bm{\gamma_{T}}$} & \parbox[bottom][1cm][c]{2cm}{\textbf{Max. OC size found}} & \textbf{$\bm{\sqrt{n}}$} \\ [1.8ex]
		%data
			\hline
			\textit{air01} & 2.4 & 8 & 16.9 \\
			\textit{cpsk} & 40.8 & 162 & 43.6 \\
			\textit{gb-coach} & 25.2 & 128 & 49.5 \\
			\textit{gb-train} & 25.6 & 111 & 50.5 \\
			\textit{montr} & 21.1 & 63 & 14.7 \\
			\textit{sncf} & 36.8 & 111 & 51.4 \\
			\textit{sncf-inter} & 17.1 & 58 & 19.1 \\
			\textit{sncf-ter} & 48.0 & 167 & 51.3\\
			\textit{zsr} & 15.0 & 57 & 15.3 \\
		\end{tabular}
		\captionof{table}{With one exception, OC diameter is less then $\sqrt{n}$ (this was expected, as \textit{montr} is the only timetable with ``geographically one dimension long'' - all other timetables span areas with more uniform shape). Note extremely low value for airline timetable - this is due to the fact that UGs of airline timetables have small-world characteristics~\cite{sommerthesis10}. Another thing we may notice is that regional timetables (\textit{cpsk}, \textit{sncf-ter}) have higher OC diameter then country-wide and inter-city timetables. We also point out that the inter-city trains in French railways decrease the average optimal connection size by one about third.}
		\label{table:tt_ocdiam}
	\end{minipage}
	\hspace{1cm}	
	\begin{minipage}{0.45\textwidth}
		\centering
		\begin{tabular}{c|c}
		%legend
			\rowcolor{tablehead}
			\textbf{Name} & \textbf{Maximal $\bm{\delta_{T}}$ found}\\
		%data
			\hline
			\textit{air01} & 34.5 \\
			\textit{cpsk} & 4.1 \\
			\textit{gb-coach} & 5.0 \\
			\textit{gb-train} & 5.8 \\
			\textit{montr} & 1.9 \\
			\textit{sncf} & 5.0 \\
			\textit{sncf-inter} & 3.0 \\
			\textit{sncf-ter} & 4.8 \\
			\textit{zsr} & 3.2 \\
		\end{tabular}
		\captionof{table}{Approximate density of the underlying graphs.}
		\label{table:ug_density}
		
		\centering
		\begin{tabular}{c|c|c}
		%legend
			\rowcolor{tablehead}
			\textbf{Name} & \textbf{Avg. degree} & \textbf{Max. degree}\\
		%data
			\hline
			\textit{air01} & 16.3 & 166 \\
			\textit{cpsk} & 2.7 & 27 \\
			\textit{gb-coach} & 2.4 & 103 \\
			\textit{gb-train} & 3.3 & 30 \\
			\textit{montr} & 1.6 & 5 \\
			\textit{sncf} & 3.0 & 27 \\
			\textit{sncf-inter} & 2.5 & 12 \\
			\textit{sncf-ter} & 2.9 & 27 \\
			\textit{zsr} & 2.5 & 12 \\
		\end{tabular}
		\captionof{table}{Average and maximal degree in the underlying graphs.}
		\label{table:ug_degs}
	\end{minipage}
    }
	\end{figure}
	
	\noindent The density is related to the \textbf{average degree} $deg_{avg}$ in the UG, since in oriented graphs:
	
	$$deg_{avg} = \frac{\displaystyle m}{\displaystyle n}$$
	
	\noindent So the average degree is a lower bound on the graph's density. Table~\ref{table:ug_degs} lists the average and maximal degrees in the underlying graphs. \\
	
	\noindent We would also assume, that the underlying graphs of each timetable will be \textbf{connected} (and even strongly connected), or at least that the largest connected component spans almost the whole graph. From the table~\ref{table:ug_conn} we may see that this assumption holds. \\

	\begin{table}[H]
		\centering
		\begin{tabular}{c|c|c|c|c|c}
		%legend
			\rowcolor{tablehead}
			 & & \multicolumn{2}{>{\columncolor{tablehead}}c|}{\textbf{Connectivity}} & \multicolumn{2}{>{\columncolor{tablehead}}c}{\textbf{Strong connectivity}} \\ \hline
			\rowcolor{tablehead}
			\textbf{Name} & $\bm{n}$ & \textbf{Connected} & \textbf{Largest comp.} & \textbf{Connected} & \textbf{Largest comp.} \\
		%data
			\hline
			\textit{air01} & 287 & \tick & 287 & \cross & 286 \\
			\textit{cpsk} & 1905 & \tick & 1905 & \cross & 1903 \\
			\textit{gb-coach} & 2448 & \cross & 2374 & \cross & 2332 \\
			\textit{gb-train} & 2555 & \tick & 2555 & \tick & 2555 \\
			\textit{montr} & 217 & \cross & 211 & \cross & 209 \\
			\textit{sncf} & 2646 & \tick & 2646 & \cross & 2594 \\
			\textit{sncf-inter} & 366 & \cross & 328 & \cross & 316 \\
			\textit{sncf-ter} & 2637 & \tick & 2637 & \cross & 2583 \\
			\textit{zsr} & 233 & \tick & 233 & \cross & 225 \\
		\end{tabular}
		\captionof{table}{Connectivity of underlying graphs.}
		\label{table:ug_conn}
	\end{table}
	
	\noindent In the previous section~\ref{sec:relwrk} we have mentioned the highway dimension~\cite{highwaydim10} as a parameter which, when being low, guarantees low query times for certain route-planning methods. Here we were interested in the highway dimension of our underlying graphs.
	
	\begin{definition}
		\textbf{Highway dimension} \\
		Highway dimension $\bm{HD(G)}$ for a directed, edge-weighted graph $G = (V, E)$ is the smallest integer $h$, such that:
        \vskip 10pt
        \indent $\forall r \in R^{+}, \; \forall u \in V, \; \exists S \subseteq B_{u, 2r}, \; |S| \leq h, \; \forall v, w \in B_{u, 2r}$: \\
        \indent \indent if $r < |P(v, w)| \leq 2r$ and $P(v, w) \subseteq B_{u, 2r}$ then $P(v, w) \cap S \neq \emptyset$
        \vskip 10pt
		\noindent where:
		\begin{itemize}
			\item $\bm{P(v, w)}$ is the \textbf{shortest path} between $v$ and $w$
			\item $\bm{B_{u, r}} = \{v \in V | \; |P(u, v)| \leq r$ or $|P(v, u)| \leq r\}$ and is called \textbf{ball} of radius $r$ centred at $u$.
		\end{itemize}
	\end{definition}

	\noindent Intuitively, a graph has a low HD, if for any $r$ we have a \emph{sparse} set of vertices $S_{r}$, such that every shortest path longer then $r$ includes a vertex from $S_{r}$. By the set being sparse, we mean that every ball of radius $\mathcal{O}(r)$ contains just a few elements of $S_{r}$.

	\begin{figure}[h!]
    	\begin{center}
        	\inputTikZ{./tikzpics/highwaydim}
		\end{center}
        \caption{\label{fig:highdemonstr} Demonstration of a definition of HD. We chose some $r$ ($r = 6$) and some vertex $v$ ($v = C$) to root the ball $B_{v, 2r}$. All the shortest paths \emph{longer} than $r$ \emph{inside} the ball have to contain a vertex from $S$ (orange vertices $C$ and $A$ in our case). The upper bound on $|S|$, considering any ball with any radius, is the required highway dimension. Note: in our case, we had to choose also $A$ to set $S$, since a shortest path from $B$ to $D$ does not include $C$.}
	\end{figure}
	
	\noindent Highway dimension is difficult to measure due to the way it is defined. We compute at least approximate value for sub-graphs of 200 nodes of the underlying graphs. Results could be seen in table~\ref{table:ug_hd}.
	
	\begin{table}[H]
		\centering
		\begin{tabular}{c|c}
		%legend
			\rowcolor{tablehead}
			\textbf{Name} & apx. $HD$ \\
		%data
			\hline
			\textit{svk-200} & \textbf{32} \\
			\textit{air01-200} & 82 \\
			\textit{cpsk-200} & 127 \\
			\textit{gb-coach-200} & 107 \\
			\textit{gb-train-200} & 147 \\
			\textit{montr-200} &  \\
			\textit{sncf-200} &  \\
			\textit{sncf-inter-200} &  \\
			\textit{sncf-ter-200} &  \\
			\textit{zsr-200} &  \\
		\end{tabular}
		\captionof{table}{Highway dimension of 200 vertex sub-graphs of the underlying graphs.}
		\label{table:ug_hd}
	\end{table}
	