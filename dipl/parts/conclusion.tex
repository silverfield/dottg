We have developed exact methods to considerably speed-up the query time for optimal connections in timetables compared to the time-dependent Dijkstra's algorithm (running in $\mathcal{O}(m + n \log n)$). Our first algorithm - \textit{USP-OR} - achieves speed-ups of up to 70 in the sub-timetable of country-wide coaches in Great Britain. However, it does so at the cost of high space consumption, requiring more that 50 times the space that is needed to represent the timetable itself. Theoretically, for real-world timetables with certain properties, this algorithm has the space complexity $\mathcal{O}(n^{2.5})$ and the average query time $\mathcal{O}(\sqrt{n})$.
	
Our second algorithm called \textit{USP-OR-A} is still 8.5 times faster then the time-dependent Dijkstra's algorithm on the dataset of British coaches (2500 stations) and at the same time, it requires about 4 times the space needed to represent the timetable. We believe the speed-up of \textit{USP-OR-A} against Dijkstra's algorithm can be even higher for bigger timetables, since its query time is under certain conditions theoretically determined as $\mathcal{O}(\sqrt{n} \log n)$, while the algorithm can handle much bigger datasets for its space complexity is essentially $\mathcal{O}(n^{1.5})$. 
	
Finally, it would be interesting to measure the query times of \textit{USP-OR-A} if we used random sampling of queries with a distribution according to the reality. Such distribution strongly favours queries concerning the most important cities which are generally part of the access node set in \textit{USP-OR-A}. As computing optimal connections between these cities is very fast (just like in \textit{USP-OR}), we could expect much better speed-ups in real-world situations.