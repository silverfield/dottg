\noindent In this thesis we studied the optimal connection problem in timetables on which we are allowed to carry out preprocessing. We formally approached the topic by clearly defining the terminology, model of the timetable and its graph representations, as well as the approach that is based on the preprocessing of the timetable. On a more practical note, we have gathered numerous real-world timetables of various type and scale and analysed their main properties. \\

\noindent In the hearth of this thesis, we have developed exact methods to considerably speed-up the query time for the optimal connections compared to the time-dependent Dijkstra's algorithm (running in $\mathcal{O}(m + n \log n)$). Our first algorithm - \textit{USP-OR} - is based on pre-computing paths, that are worth to follow (the so called underlying shortest paths). The method achieves speed-ups of up to 113 on the weekly timetable of US domestic flights, requiring about 10 times the memory needed to represent the timetable itself. A noticeable speed-up (up to 70) was also reached for the sub-timetable of country-wide coaches in Great Britain. However, here the space consumption was much higher (a factor of more than 50), which was also the case with other timetables of bigger scale. We conclude that the algorithm is suitable for smaller-size timetables with very short connections, such as airline timetables. With most bus/train timetables this method has a space complexity $\mathcal{O}(\tau n^{2.5})$ and an average query time $\mathcal{O}(\tau \sqrt{n})$.
	
Our second algorithm called \textit{USP-OR-A} computes a small set of important access stations and an additional information for optimal travelling between these stations. It is much less space demanding, however still more than 8 times faster then the time-dependent Dijkstra's algorithm on the dataset of British country-wide coaches (about 2500 stations) and about 6 times faster on the dataset of French railways (also about 2500 stations). The size of the preprocessed data was no more than 7.5 times the size of the timetable for any of our datasets. Under certain conditions, this algorithm has an average query time $\mathcal{O}(\tau \sqrt{n} \log n)$ and the space complexity $\mathcal{O}(\tau n^{1.5})$. This method works well on timetables with sparse underlying graphs that contain small sets of important transit-like stations. We showed that finding these sets is NP hard problem in general, however, the heuristics we developed for this purpose seem to behave reasonably well.
	
With respect to \textit{USP-OR-A}'s query times, we would like to note that we measured the query times using a completely uniform distribution of queries, which is not very realistic scenario. A real-world distribution is much different in that it strongly favours queries concerning the most important cities. We observed that such cities were generally part of the access node sets in \textit{USP-OR-A}~\footnote{E.g. the stations like Gare de Lyon were \textit{always} part of the access node set.}. As computing optimal connections between these cities is very fast (just like in \textit{USP-OR}), we could expect much better speed-ups in real-world situations. \\

\noindent From other contribution of this thesis we mention the application \textit{TTBlazer} for timetable analysis and performance tests that we developed for the purpose of this thesis, as well as the experiment in which we tried to train a neural network to answer optimal connection queries. This approach turned out as not working good enough, leading to our belief that the problem in question is too demanding for a neural network to learn. \\

\noindent To conclude, we feel this thesis provides useful techniques, results and information in general that might be of interest when designing a large-scale timetable information systems.