In this section, we provide most of the definitions and terminology used throughout the thesis. 

	\subsection{Objects}

	First, we will formalize the notion of a timetable and its derived graph forms, the underlying graph and terms related to these objects.
	
	\begin{definition}
        \textbf{Timetable (TT)} \\
        A timetable is a set $\bm{T} = \{(x, y, p, q)| \; p, q \in \mathcal{N}, \; p < q \}$.
        \begin{itemize}
        	\item Elements of T (the 4-tuples) are called \textbf{elementary connections}. For an elementary connection $e = (x, y, p, q)$:
        	\begin{itemize}
        		\item $\bm{from(e)} = x$ is the \textbf{departure city}
        		\item $\bm{to(e)} = y$ is the \textbf{arrival/destination city}
        		\item $\bm{dep(e)} = p$ is the \textbf{departure time}
        		\item $\bm{arr(e)} = q$ is the \textbf{arrival time}
        	\end{itemize}
        	\item The set of all \textbf{cities} (stations) will be denoted as $\bm{ct_{T}} = \{x| \; (x, y, p, q) \in T \; or \; (y, x, p, q) \in T\}$ and the number of cities as $n_{T}$. 
        	\item Pairs $(x, p)$ or $(y, q)$ such that $(x, y, p, q) \in T$ form the set of \textbf{events} $\bm{ev_{T}}$. The set of events in a specific city $x$ is $\bm{ev_{T}(x)} = \{(x, t)| \; (x, y, t, q) \in T \; or \; (y, x, p, t) \in T\}$
        	\item Let $\bm{tlow_{T}} = \min_{e \in T} dep(e)$ and $\bm{thigh_{T}} = \max_{e \in T} arr(e)$. The value $\bm{tr_{T}} = thigh_{T} - tlow_{T}$ is called the \textbf{time range} of the timetable.
        	\item \textbf{Height} of the timetable is the average number of events in a city: 
        	$$\bm{h_{T}} = \frac{|ev_{T}|}{n_{T}}$$
        \end{itemize}
    \end{definition}
    
    \noindent Let us describe some the defined terms more informally. An elementary connection corresponds to moving from one stop to the next one, e.g. with a bus (thus we disregard the notion of \textit{lines}, i.e. getting on and off). Note that we express time as an integer - throughout this paper, this integer will represent the minutes elapsed from the time 00:00 of the first day. Thus we may take the liberty of talking about time in integer or \textit{days hh:mm} format, as convenient at the moment. Lastly, an event simply represent an arrival or departure of a e.g. train at some station. The remaining terms should be clear enough.
    
    \begin{table}[h!]
    	\centering
        \begin{tabular}{c|c|c|c}
        %legend
            \rowcolor{tablehead}
            \multicolumn{2}{>{\columncolor{tablehead}}c|}{\textbf{Place}} & \multicolumn{2}{>{\columncolor{tablehead}}c}{\textbf{Time}} \\
			\hline
           	\rowcolor{tablehead}
            \textbf{From} & \textbf{To} & \textbf{Departure} & \textbf{Arrival} \\
		%data
			\hline
            \textcolor{city-clr}{A} & \textcolor{city-clr}{B} & 10:00 & 10:45 \\
			\textcolor{city-clr}{B} & \textcolor{city-clr}{C} & 11:00 & 11:30 \\
			\textcolor{city-clr}{B} & \textcolor{city-clr}{C} & 11:30 & 12:10 \\
			\textcolor{city-clr}{B} & \textcolor{city-clr}{A} & 11:20 & 12:30 \\
			\textcolor{city-clr}{C} & \textcolor{city-clr}{A} & 11:45 & 12:15 \\
		\end{tabular}
		\caption{\label{table:tt} An example of a timetable - the set of elementary connections (between pairs of \textcolor{city-clr}{cities}). An example of an event is a pair (A, 10:00), when some el. connection departs from $A$.}
	\end{table}
	
	\noindent Following is a definition of a connection.
    
   	\begin{definition}
        \textbf{Connection} \\
        A connection from $a$ to $b$ is a sequence of elementary connections $\bm{c} = (e_{1}, e_{2}, ..., e_{k}), k \geq 1$, such that $from(e_{1}) = a$, $to(e_{k}) = b$ and $\forall i \in \{2, ..., k\}: (to(e_{i}) = from(e_{i - 1}), \; arr(e_{i}) \geq dep(e_{i - 1})$.
        \begin{itemize}
			\item Connection \textbf{starts} at the departure time $\bm{start(c)} = dep(e_{1})$ and \textbf{ends} at the arrival time $\bm{end(c)} = arr(e_{k})$.
			\item We also extend $\bm{from(c)} = from(e_{1})$ and $\bm{to(c)} = to(e_{k})$
	        \item \textbf{Length} of the connection is $\bm{len(c)} = end(c) - start(c)$
	        \item \textbf{Size} of the connection is $\bm{size(c)} = k$~\footnote{We will use similar terminology when talking about paths - the \textit{size} is the number of vertices (hops) in the path while the \textit{length} refers to the actual distance (sum of weights of the edges in the path).}
	        \item We will denote the set of \textbf{all connections} from $a$ to $b$ in a timetable $T$ as $\bm{C_{T}(a, b)}$. We also define $\bm{C_{T}} = \cup_{a, b} C_{T}(a, b)$
        \end{itemize}
    \end{definition}
    
    \noindent So we understand connection as a (valid) sequence of elementary connections. 
    
    \begin{figure}[h!]
        \begin{center}
			\inputTikZ{./tikzpics/connection}
        \end{center}
		\caption{\label{pic:conn} A valid connection made out of \textcolor{elcon-clr}{elementary connections} (and \textcolor{waiting-clr}{waiting}, which is implicit).}
	\end{figure}
	
	\noindent Next, we continue with the underlying graph - a graph representing basically the map on top of which the timetable operates.
	
	\begin{definition}
        \textbf{Underlying graph (UG graph)} \\
        The underlying graph of a timetable $T$, denoted $\bm{ug_{T}}$, is an oriented graph $(V, E)$, where $V$ is the set of all timetable cities and $E = \{(x, y)|\; \exists (x, y, p, q) \in T\}$
        \begin{itemize}
        	\item By $\bm{m_{T}}$ we will denote the number of arcs in the UG
        \end{itemize}
    \end{definition}
    
    \noindent Note, that we do not specify the weights of the edges in the underlying graph - they will be specified based on the current usage of the UG. Most of the time, however, if we work with a weighted UG, the weight of an arc will be the length of the shortest elementary connection on that arc. More specifically, $w(x, y) = \min_{(x, y, p, q) \in T} (q - p) \; \forall (x, y) \in E(ug_{T})$. Such weighted UG will be called \textbf{optimistic} (denoted $\bm{ug_{T}^{opt}}$).
    
    \begin{figure}[h!]
        \begin{center}
			\inputTikZ{./tikzpics/ug}
        \end{center}
		\caption{\label{fig:ug} An optimistic underlying graph of the timetable~\ref{table:tt}. The nodes are the \textcolor{city-clr}{cities} of the timetable.}
	\end{figure}
	
	\noindent If we want to represent the timetable by a graph, there are two most common options~\cite{timetablemodelsalgs07} - the time-expanded and time-dependent graph.
	
	\begin{definition}
		\textbf{Time-expanded graph (TE graph)} \\
        Let $T$ be a timetable. Time-expanded graph from $T$, denoted $\bm{te_{T}}$, is an oriented graph $(V, E)$ whose vertices correspond to events of $T$, that is $V = \{[x, t]| \; (x, t) \in ev_{T}\}$. The edges of $G$ are of two types
        \begin{enumerate}
            \item $([x, p], [y, q]) \; \forall (x, y, p, q) \in T$ - the so called \textbf{connection edges}
            \item $([x, p], [x, q]) \; [x, p], [x, q] \in V, \; p < q$ and $\not \exists [x, r] \in V\colon p < r < q$. - the so called \textbf{waiting edges}
		\end{enumerate}
        Weight of the edge $([x, p], [y, q])$ is $w([x, p], [y, q]) = q - p$.
	\end{definition}
	
	\noindent Informally, an edge in TE graph represent either the travelling with an elementary connection or waiting for the next event in the same city. Also, the time range and height of a timetable could be easily illustrated on the TE graph (see picture~\ref{pic:te}). 
	
	\begin{figure}[h!]
	    \begin{center}
			\inputTikZ{./tikzpics/te}
	    \end{center}
    	\caption{\label{pic:te} Time-expanded graph of the timetable~\ref{table:tt}. Nodes represent the \textcolor{event-clr}{events}. There are \textcolor{elcon-clr}{connection} and \textcolor{waiting-clr}{waiting} edges (dashed). The time range is 2h:30m and the height is 4 (as there are 4 events in city B).}
	\end{figure}
        
    \begin{definition}
		\textbf{Time-dependent graph (TD graph)}\\
        Let $T$ be a timetable. Time-dependent graph from $T$, denoted $\bm{td_{T}}$, is an oriented graph $(V, E)$ whose vertices are the timetable cities and $E = \{(x, y)| \; \exists (x, y, p, q) \in T\}$. Furthermore, the weight of an edge $(x, y) \in E$ is a piece-wise linear function $w(x, y) = f_{x, y}(t) = q - t$ where $q$ is:
        \begin{itemize}
        	\item $\min \{arr(e) | e \in T, \; dep(e) \geq t\}$
        	\item $\infty$, if $dep(e) < t \; \forall e \in T$
        \end{itemize}
	\end{definition}
	
	\noindent Intuitively, the TD graph is simply the UG graph where each arc carries a function specifying the traversal time of that arc at any time. For an example, see picture~\ref{pic:piece}: The latest point of every linear segment is called the \textbf{interpolation point} and it corresponds to an elementary connection (its coordinates are $dep(e), len(e)$ for corresponding el. connection $e$). Note that a list of all interpolation points fully defines the piece-wise linear function.
	
	The algorithms in this thesis use almost exclusively the TD graphs, mainly because they are less space consuming. Also, time-dependent Dijkstra searches are a bit faster on TD graphs, because the search space that has to be explored is smaller. On the other hand, TE graphs are more flexible when we need to take additional search parameters into consideration (like transfers, travel costs). Since we will not talk about these, TD graphs are more suitable. 
	
	\begin{figure}[htb]
	\centering
	\makebox[0pt][c]{
	\begin{minipage}{.45\textwidth}
		\centering
		\inputTikZ{./tikzpics/td}
		\captionof{figure}{Time-dependent graph of the timetable~\ref{table:tt}. The nodes are the \textcolor{city-clr}{cities}.}
		\label{pic:td}
	\end{minipage}
	\hspace{1cm}
	\begin{minipage}{.45\textwidth}
		\centering
		\inputTikZ{./tikzpics/tdpiece}
		\captionof{figure}{Piece-wise linear function - traversal times for the arc $(B, C)$. The \textcolor{red}{highlighted} points are the interpolation points.}
		\label{pic:piece}
	\end{minipage}
	}
	\end{figure}
	
	\noindent To sum up, there are four main types of objects we will be working with:
	\begin{itemize}
		\item Timetable (TT)
		\item Underlying graph (UG)
		\item Time-expanded graph (TE)
		\item Time-dependent graph (TD)
	\end{itemize}
	\hspace*{\fill}
		
	\noindent \textit{Note:} Throughout this paper, we will relax a bit the notation and leave out subscripts (e.g. $ug_{T}~\rightarrow~ug$, $n_{T}~\rightarrow~n$, etc.) in situations, where the context is clear enough. 

\subsection{Earliest arrival and optimal connection}

	Now we would like to formulate the main problems this thesis deals with.
	
	\begin{definition}
		\textbf{Earliest arrival problem (EAP)}\\
        Given a timetable $T$, departure city $x$, destination city $y$ and a departure time $t$, the task is to determine $\bm{t_{(x, t, y)}^{*}} = \min_{c \in C_{T}(x, y)} \{t + len(c)| \; start(c) \geq t\}$. 
        \begin{itemize}
        	\item We will refer to the tuple $\bm{(x, t, y)}$ as an \textbf{EAP instance}, or an \textbf{EAP query}
        	\item The time $\bm{t_{(x, t, y)}^{*}}$ is called the \textbf{earliest arrival (EA)} for the given EAP instance
        \end{itemize}
	\end{definition}
	
	\noindent A bit more difficult version of this problem is one, where we require to actually output the connection ending at time given by EA.

	\begin{definition}
		\textbf{Optimal connection problem (OCP)}\\
        Given a timetable $T$, departure city $x$, destination city $y$ and a departure time $t$, the task is to find the \textbf{optimal connection (OC)} $\bm{c_{(a, t, b)}^{*}} = argmin_{c \in C_{T}(a, b)} \{t + len(c)| \; start(c) \geq t\}$. 
	\end{definition}
	
	\noindent The instance/query in case of the optimal connection problem has the same form as EAP query. Also, note that the OCP is at least as hard to solve as EAP since having the optimal connection implies the optimal (earliest) arrival time. In order to avoid technical issues (e.g. in the definition), we may assume that the optimal connection is unique (i.e., there is not a different connection with the same end time). However, we consider any connection which arrives at time $t_{(x, t, y)}^{*}$ to be optimal for the given query.
	
	\begin{example}
		Consider our timetable from table~\ref{table:tt}. For the EAP instance (B, 10:45, A), the earliest arrival (EA) is 12:15 and the optimal connection (OC) is ((B, C, 11:00, 11:30), (C, A, 11:45, 12:15)), as could be easily seen from picture~\ref{pic:ea} of the TE graph.
	\end{example}

	\begin{figure}[h!]
	    \begin{center}
			\inputTikZ{./tikzpics/ea}
	    \end{center}
    	\caption{\label{pic:ea} Optimal connection and earliest arrival time are marked in \textcolor{orange}{\textbf{bold}}.}
	\end{figure}
	
\subsection{(Distance) Oracles}

	The term \textit{distance oracle} was first coined in 2001 by Thorup and Zwick~\cite{apxdo05}, when talking about quick shortest path (or distance) computations on graphs. One approach to this problem is to pre-compute some information on the graph to speed-up answering of the queries. The paper of Thorup and Zwick was dealing with trade-offs among the time complexity of the pre-computation, the amount of pre-computed information, the speed-up in query times and the accuracy of the answers. Since the pre-computed data structure is something that helps us answer the queries more efficiently, it resembles an oracle, thus the term distance oracle.

	In this thesis, we will discuss methods that behave the same way, but deal with the earliest arrival problem (or optimal connection problem) - there is some pre-processing of the timetable with a resulting data structure that speeds up answering subsequent queries. To formalize this a little more, we will refer to this kind of methods as \textbf{oracle based methods}. For such a method $m$, we are interested mainly in its four parameters:
	\begin{itemize}
		\item \textbf{Preprocessing time} ($\bm{prep(m)}$) - the time complexity of the pre-computation
		\item \textbf{Preprocessed space} ($\bm{size(m)}$) - the space complexity of the pre-computed data structure (the so called \textbf{oracle})
		\item \textbf{Query time} ($\bm{qtime(m)}$) - the time complexity of answering a single query
		\item \textbf{Stretch} ($\bm{stretch(m)}$) - the worst-case ratio against the optimal value of earliest arrival (the lower, the better)
	\end{itemize}
	\hspace*{\fill}
	
	\begin{figure}[h!]
	    \begin{center}
			\inputTikZ{./tikzpics/oracle}
	    \end{center}
    	\caption{\label{pic:oracle} Principle of oracle based methods - we preprocess the timetable, creating a structure that helps us speed-up the answers to queries for optimal connection.}
	\end{figure}
	
	\noindent The preprocessing time is probably the least critical resource. A reasonable polynomial should bind its time complexity, depending on the computational power of the user and the scale of the timetable. The size of the preprocessed oracle is much more important - in the optimal case, it should be bound by the space complexity of the timetable itself. Optimality of the query time depends on which problem we are solving. If we query for the whole optimal connection, we have to count with a time complexity at least proportional to the diameter of the underlying graph (as connections could be that long, or even longer). If we require only the EA value as an output, much better speed-ups could be expected. The stretch should be of course as low as possible.
	
\subsection{Dijkstra's algorithm} 
\label{subsec:dijkstra}
	
	\noindent Throughout this thesis, we will often use Dijkstra's algorithm and its modifications both as a part of our algorithms and as a reference point against which we will compare the performance of our methods. This is a common practice. Researchers working on methods answering distance or shortest path queries in road networks commonly use the term \textit{speed-up}, i.e. \textit{how many times faster} is their algorithm against the Dijkstra's algorithm.
	
	Dijkstra's algorithm is originally an algorithm that looks for shortest paths in weighted oriented graphs. It was published by E. W. Dijkstra in 1959~\cite{dijkstra59} and we will not explain it at this place, as the algorithm is very well explained at many other places (e.g.~\cite{dijkstraexp}). For a good summary of Dijkstra's algorithm related implementations and publications see~\cite{sommerthesis10}.
	
	As our task is to compute earliest arrivals or optimal connections instead of distances and shortest paths, our ``reference point'' will be a slightly modified Dijkstra's algorithm called \textbf{time-dependent Dijkstra's algorithm}~\cite{tdroute09} (or \textbf{TD Dijkstra} for short). The algorithm is run on a time-dependent graph and works just like the ordinary Dijkstra's algorithm, except that the weight of each arc $(x, y)$ is determined for the time $t$ at which we had settled vertex $x$.
	
	If we assume that the evaluation of an arc by the cost function of the TD graph is implemented in constant time, the running time of the TD Dijkstra is $\mathcal{O}(n^{2})$, just like the normal Dijkstra's algorithm. On sparse graphs, this bound can be improved using a quick data structure to determine the next node we settle. A good option is a priority queue implemented as a \textit{Fibonacci heap}, which implements deletion in $\mathcal{O}(\log n)$ and all other operations in constant amortized time~\cite{sommerthesis10}. This yields the running time of TD Dijkstra $\mathcal{O}(n \log n + m)$.
	
	We may therefore introduce a fifth parameter of our oracle based methods, the speed-up:
	
	\begin{definition}
		\textbf{Speed-up ($\bm{spd(m)}$)}\\
        A speed-up of an oracle based method $m$ is the ratio $\frac{\displaystyle qtime_{avg}(TD Dijkstra)}{\displaystyle qtime_{avg}(m)}$ where $qtime_{avg}(m')$ is the average query time of the respective oracle based method $m'$~\footnote{Note that we may also consider the TD Dijkstra algorithm to be an oracle based method - it just happens that it does not require any preprocessing.}.
	\end{definition}
	
	\noindent The definition is rather loose in the sense that we may refer to a concrete speed-up of the method on a concrete dataset or a general, theoretical speed-up expressed as a function of the size of input.