\noindent \textbf{Timetable} is simply a set of elementary connections, thus the format is:
\begin{itemize}
	\item number of el. connections
	\item the list of all el. connections (one per line, format ``\textit{FROM TO DEP-DAY DEP-TIME ARR-DAY ARR-TIME}'')
\end{itemize}
\hspace*{\fill}

\begin{lstlisting}[caption={TT file format.}]
7                      //number of elementary connections
A B 0 10:00 0 10:45    //el. connection
A B 0 11:00 0 11:45
A B 0 12:00 0 12:45
A C 0 09:30 0 10:00
A C 0 10:15 0 10:45
C D 0 11:00 0 11:30
C D 0 13:00 0 13:30
\end{lstlisting}

\noindent \textbf{Underlying graph} is basically an oriented graph, with some optional parameters. The format is the following:
\begin{itemize}
	\item number of cities
	\item number of arcs
	\item the list of all cities  (one per line)
	\begin{itemize}
		\item optional coordinates (otherwise null)
	\end{itemize}
	\item the list of all arcs (one per line, format ``\textit{FROM TO}'')
	\begin{itemize}
		\item optional length (otherwise null)
		\item optional list of lines operating on that arc (otherwise null)
	\end{itemize}
\end{itemize}
\hspace*{\fill}
			
\begin{lstlisting}[caption={UG file format.}]
4                                      //number of cities
5                                      //number of arcs
A 45 32                                //name of the city, optional coordinates
B null
C 56 34
D null
A B 57 Northern                        //arc, optional length and list of lines 
A C null Picadilly Victoria            
C B 45 Circle Jubilee Picadilly
C D 32 null                          
D A null null
\end{lstlisting}

\noindent \textbf{Time-expanded graph} is simply an oriented weighted graph, with nodes being the events and arcs being the elementary connections or waiting edges:
\begin{itemize}
	\item number of nodes (i.e. events)
	\item number of arcs (el. connections + waiting)
	\item the list of all events (in the format ``CITY DAY TIME'')
	\item the list of all arcs (in the format ``FROM-EVENT TO-EVENT'')
\end{itemize}
\hspace*{\fill}

\begin{lstlisting}[caption={TE file format.}]
5                       //number of events
15                      //number of arcs
A 0 13:30               //event
A 0 14:00
B 0 13:45
B 0 15:00
C 0 14:15
A 0 13:30 A 0 14:00     //waiting arc
A 0 13:30 B 0 13:45     //el. connection arc
A 0 14:00 B 0 15:00
A 0 13:30 B 0 15:00
C 0 14:15 B 0 15:00
...
\end{lstlisting}

\noindent \textbf{Time-dependent graph} is an oriented graph with a function on the arc specifying the arc's traversal time at any moment. In timetable networks this function is piece-wise linear and it is fully represented by the list of its interpolation points. Thus the TD file format:
\begin{itemize}
	\item number of cities
	\item number of arcs
	\item the list of all cities  (one per line)
	\begin{itemize}
		\item optional coordinates (otherwise null)
	\end{itemize}
	\item the list of all arcs (one per line). Arc has the format ``\textit{FROM TO INT-POINTS}'' where \textit{INT-POINTS} is a list of interpolation points~\footnote{An interpolation point is described by a triple ``\textit{DAY TIME MINUTES}'', where \textit{MINUTES} are the traversal time}, see the listing~\ref{lst:tdformat} for an example.
\end{itemize}
\hspace*{\fill}

\begin{lstlisting}[caption={TD file format.}, label={lst:tdformat}]
4                                           //number of stations
5                                           //number of arcs
A 0 0                                       //name of the city, optional coordinates
B 4 4
C null
D 12 0
A B (0 13:30 45) (0 14:00 40)               //arc and the list of interpolation points
A C (1 14:15 10)
C B (0 15:00 20)
C D (2 10:00 70)
D A (1 17:20 35) (1 18:00 40) (1 18:50 35)
...
\end{lstlisting}