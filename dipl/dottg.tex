%---------------------------------------------------------------------
%   documentclass
%---------------------------------------------------------------------
\documentclass[a4paper]{article}

%---------------------------------------------------------------------
%   packages
%---------------------------------------------------------------------

\usepackage[english]{babel}
\usepackage[enc=utf8]{hrlatex}
\usepackage[T1]{fontenc} %pekne makcene

\usepackage{algorithmic}
\usepackage{algorithm}
\usepackage{amsmath}
\usepackage{amsthm}
\usepackage[titletoc, page]{appendix}
\usepackage{booktabs}% http://ctan.org/pkg/booktabs
\usepackage{bm}
\usepackage{caption}
\usepackage{cite} %bibtex
\usepackage{color} %for \definecolor
\usepackage{colortbl} %for \rowcolor command
\usepackage{courier}
\usepackage{enumitem}
\usepackage{eucal} %for nice letters like \mathcal{A}
\usepackage{floatflt} %to have tables and text beside
\usepackage{hyperref} %odkazy
\usepackage{listings}
\usepackage{lmodern} %spolu s T1 smooth font!
\usepackage{lscape}
\usepackage{mathtools}
\usepackage{multirow}% http://ctan.org/pkg/multirow
\usepackage{pdfpages}
\usepackage{pgfplots}
\usepackage{pifont} %for ticks (check symbols)
\usepackage{scalefnt}
\usepackage{setspace}
\usepackage[rightcaption]{sidecap}
\usepackage{subcaption}
\usepackage{tikz}
\usepackage{titlesec} %section titles font size change
\usepackage{xcolor} %for \colorlet

\usetikzlibrary{shapes,fit,calc,shadows,plotmarks}
\usetikzlibrary{decorations.pathreplacing}

%---------------------------------------------------------------------
%   margins
%---------------------------------------------------------------------
\oddsidemargin 0.3in
\evensidemargin 0.3in
\textwidth 6in
\textheight 23.5cm
\topmargin -0.35in

\linespread{1.2}
\renewcommand{\arraystretch}{1.1} %spacing of table rows

%---------------------------------------------------------------------
%   various settings
%---------------------------------------------------------------------
\newif\ifmine % introduce a switch for draft vs. final document
\minetrue

\newif\iffinal % introduce a switch for draft vs. final document
\finaltrue % use this to compile the final document
\iffinal
  \newcommand{\inputTikZ}[1]{%
    \input{#1.tikz}%
  }
\else
  \newcommand{\inputTikZ}[1]{%
    \beginpgfgraphicnamed{#1-external}%
    \input{#1.tikz}%
    \endpgfgraphicnamed%
  }
\fi

\setlist{nolistsep} %so that lists have normal spacing

\titleformat{\section}{\LARGE\bfseries}{\thesection}{1em}{} %section titles
\titleformat{\subsection}{\Large\bfseries}{\thesubsection}{1em}{} %subsection titles

\definecolor{tablehead}{RGB}{238,233,233} %nice smooth grey
\definecolor{algcolor}{RGB}{0,0,0}
\definecolor{inalgcolor}{RGB}{0,0,0}
\definecolor{lstcolor}{RGB}{238,233,233}

\setlength{\parindent}{0pt} %we don't need no indentation

\graphicspath{{./pics/}} %picture dir

\lstset{ %
    language=Octave,                % choose the language of the code
    basicstyle=\footnotesize\ttfamily,       % the size of the fonts that are used for the code
    numbers=left,                   % where to put the line-numbers
    numberstyle=\footnotesize,      % the size of the fonts that are used for the line-numbers
    stepnumber=1,                   % the step between two line-numbers. If it's 1 each line will be numbered
    numbersep=5pt,                  % how far the line-numbers are from the code
    backgroundcolor=\color{lstcolor},   % choose the background color. You must add \usepackage{color}
    showspaces=false,               % show spaces adding particular underscores
    showstringspaces=false,         % underline spaces within strings
    showtabs=false,                 % show tabs within strings adding particular underscores
    frame=single,	                % adds a frame around the code
    tabsize=2,	                    % sets default tabsize to 2 spaces
    captionpos=b,                   % sets the caption-position to bottom
    breaklines=true,                % sets automatic line breaking
    breakatwhitespace=false,        % sets if automatic breaks should only happen at whitespace
    title=\lstname,                 % show the filename of files included with \lstinputlisting; also try caption instead of title
    escapeinside={\%*}{*)},          % if you want to add a comment within your code
%    morekeywords={*,...}            % if you want to add more keywords to the set
	deletekeywords={all, null, length, path, function}
}

\colorlet{city-clr}{green!70!black}
\colorlet{elcon-clr}{red}
\colorlet{event-clr}{blue}
\colorlet{waiting-clr}{olive}
\colorlet{cmt-clr}{gray}
\colorlet{oracle-clr}{orange!30}
\colorlet{algsec-clr}{black!50!red}

%---------------------------------------------------------------------
%   environments
%---------------------------------------------------------------------
\renewenvironment{abstract}[1]
{
	\Large
	\begin{center}
		\textbf{#1}
	\end{center}
	
	\normalsize
	
	\addtolength{\leftskip}{1in}
	\addtolength{\rightskip}{1in}
	\setlength{\parindent}{0in}
}
{
}

\newenvironment{itemizesp}
{
    \begin{itemize}
}
{
    \end{itemize}
}

\newcommand{\cmt}[1]{{\color{cmt-clr} \hspace*{1cm} \# \textit{#1}}}
\newcommand{\algsec}[1]{\textcolor{algsec-clr}{\textbf{\underline{#1}}}} 
\newcommand{\deftoken}{\boldmath{$\mathcal{DEFINITION}$}}
\newcommand{\restoken}{\boldmath{$\mathcal{RESULT}$}}
\newcommand{\dotoken}{\boldmath{$\mathcal{DO METHOD}$}}
\newcommand{\textbff}[1]{{\large \textbf{#1}}}
\newcommand{\tick}{\ding{52}}
\newcommand{\cross}{\ding{55}}

\numberwithin{algorithm}{section}
\numberwithin{figure}{section}
\numberwithin{table}{section}
\numberwithin{equation}{section}

\newtheorem{definition}{Definition}[section]
\newtheorem{example}{Example}[section]
\newtheorem{theorem}{Theorem}[section]
\newtheorem{lemma}{Lemma}[section]
\newtheorem{observation}{Observation}[section]

\interfootnotelinepenalty=10000

%---------------------------------------------------------------------
%   magic code
%---------------------------------------------------------------------
% Here it is: the code that adjusts justification and spacing around caption.
\makeatletter
% http://www.texnik.de/floats/caption.phtml
% This does spacing around caption.
\setlength{\abovecaptionskip}{6pt}   % 0.5cm as an example
\setlength{\belowcaptionskip}{6pt}   % 0.5cm as an example
% This does justification (left) of caption.
\long\def\@makecaption#1#2{%
  \vskip\abovecaptionskip
  \sbox\@tempboxa{#1: #2}%
  \ifdim \wd\@tempboxa >\hsize
    #1: #2\par
  \else
    \global \@minipagefalse
    \hb@xt@\hsize{\box\@tempboxa\hfil}%
  \fi
  \vskip\belowcaptionskip}
\makeatother

%---------------------------------------------------------------------
%   document
%---------------------------------------------------------------------
\begin{document}
%---------------------------------------------------------------------
%   FRONTMATTER ------------------------------------------------------
%---------------------------------------------------------------------
    %\frontmatter
    \setlength{\parindent}{0pt}
    \pagestyle{empty}
%    \setcounter{page}{200} %TODO_FINAL - change in final
% also references - square brackets
% footnotes - all with capital, bodka na konci
% dots after captions
% figure - picture
% search for [?]
    \noindent

    %---------------------------------------------------------------------
    %   title page
    %---------------------------------------------------------------------
    \begin{center}
        \begin{minipage}{0.25\textwidth} \includegraphics[width=28mm]{logouk.png} \end{minipage}
        \begin{minipage}{0.74\textwidth}
        \textbf{\large\sc
            Department of Computer Science, \\
            Faculty of Mathematics, Physics and Informatics, \\
            Comenius University in Bratislava
        }
        \end{minipage}

        \vskip 6cm

        \begin{center} \line(1,0){350} \end{center}
        {\LARGE\sc Distance oracles for timetable graphs } \\
        \large{(Master thesis)}
        \vskip 0.5cm
        \textbf{\large bc. František Hajnovič}
        \begin{center} \line(1,0){350} \end{center}

        \vfill
    \end{center}

	\textbf{Study program}: Computer science \\
    \textbf{Branch of study}: 2508 Informatics \\
    \textbf{Supervisor}: doc. RNDr. Rastislav Královič, PhD.   \hfill Bratislava 2013

    \pagebreak
    
    %---------------------------------------------------------------------
    %   empty page
    %---------------------------------------------------------------------
    
	\thispagestyle{empty}
	\mbox{}
    \pagebreak

    %---------------------------------------------------------------------
    %   assignment
    %---------------------------------------------------------------------
	\includepdf[pages={1}]{../assignment-eng.pdf}
    
    \pagebreak
    
    %---------------------------------------------------------------------
    %   assignment
    %---------------------------------------------------------------------
	\includepdf[pages={1}]{../assignment-sk.pdf}
    
    \pagebreak

    %---------------------------------------------------------------------
    %   declaration of honesty
    %---------------------------------------------------------------------
    {~}\vfill

    I hereby declare that I wrote this thesis by myself, only with the help of the referenced literature, under the careful supervision of my thesis advisor.
    \vskip 1cm
    \hfill ................................

    \pagebreak

    %---------------------------------------------------------------------
    %   acknowledgements
    %---------------------------------------------------------------------
    \section*{Acknowledgements}
    I would like to thank very much to my supervisor Rastislav Královič for valuable remarks, useful advices and consultations that helped me stay on the right path during my work on this thesis.
    
    \hspace{40pt} I am also grateful for the support of my family during my studies and the work on this thesis.

	\vspace{1cm}    
    
    \hspace{\fill} \textit{František Hajnovič}

    \pagebreak

    %---------------------------------------------------------------------
    %   abstract
    %---------------------------------------------------------------------
    \begin{abstract}{Abstract}
		Queries for optimal connection in timetables can be answered by running Dijkstra's algorithm on an appropriate graph. However, in certain scenarios this approach is not fast enough. In this thesis we introduce methods with much better query time than that of the efficiently implemented Dijkstra's algorithm. \\

		Our first method called {\it USP-OR} is based on pre-computing paths, that are worth to follow. This method achieves speed-ups of up to 70, although at the cost of high amount of preprocessed data. Our second algorithm computes a small set of important stations and additional information for optimal travelling between these stations. Named {\it USP-OR-A}, this method is much less space consuming but still more than 8 times faster than the Dijkstra's algorithm on some of the real-world datasets. \\
    
        Other contributions of this thesis are 

		Key words: \textbf{optimal connection}, \textbf{timetable}, \textbf{Dijkstra's algorithm}, \textbf{Distance oracles}, \textbf{underlying shortest paths}
	\end{abstract}	

    \begin{abstract}{Abstrakt}
        V tejto práci sa zaoberáme hľadaním optimálnych spojení v cestovných poriadkoch, na ktorých sme si predpočítali určité informácie. Na základe analýzy reálnych cestovných poriadkov sme vyvinuli exaktné metódy, ktoré na dotaz na optimálne spojenie odpovedajú podstatne rýchlejšie ako časovo závislá implementácia Dijkstrovho algoritmu využívajúca prioritnú frontu na základe Fibonacciho haldy. Presnejšie, náš algoritmus \textit{USP-OR-A} s priestorovou zložitosťou $\mathcal{O}(n^{1.5})$ dosahuje časovú zložitosť odpovede na dotaz $\mathcal{O}(\sqrt{n} \log n)$, prekonávajúc časovo závislý Dijkstrov algoritmus takmer 7 krát v našom najväčšom cestovnom poriadku. \\

		Kľúčové slová: \textbf{optimálne spojenie}, \textbf{cestovný poriadok}, \textbf{Dijkstrov algoritmus}, \textbf{Dištančné orákulá}, \textbf{podkladové najkratšie cesty}
	\end{abstract}	
	
    \pagebreak

    %---------------------------------------------------------------------
    %   contents
    %---------------------------------------------------------------------

    \tableofcontents

    \pagebreak

%---------------------------------------------------------------------
%   MAINMATTER  ------------------------------------------------------
%---------------------------------------------------------------------
    %\mainmatter
    \pagestyle{plain}
    \setcounter{page}{1}
    \setlength{\parindent}{40pt}

    %---------------------------------------------------------------------
    %   introduction
    %---------------------------------------------------------------------
    \section{Introduction}
    \label{sec:intro}
    \input parts/introduction.tex
    \pagebreak
   
    %---------------------------------------------------------------------
    %   preliminaries
    %---------------------------------------------------------------------
    \section{Preliminaries}
    \label{sec:prel}
    \input parts/preliminaries.tex
    \pagebreak
    
    %---------------------------------------------------------------------
    %   related work
    %---------------------------------------------------------------------
    \section{Related work}
    \label{sec:relwrk}
    \input parts/relatedwrk.tex
    \pagebreak
    
    %---------------------------------------------------------------------
    %   data
    %---------------------------------------------------------------------
    \section{Data \& analysis}
    \label{sec:data}
    \input parts/data.tex
    \pagebreak
    
    %---------------------------------------------------------------------
    %   usp
    %---------------------------------------------------------------------
    \section{Underlying shortest paths}
    \label{sec:usp}
    \input parts/usp.tex
    \pagebreak
    
    %---------------------------------------------------------------------
    %   neural
    %---------------------------------------------------------------------
    \section{Neural network approach}
    \label{sec:neural}
    \input parts/neural.tex
    \pagebreak
    
    %---------------------------------------------------------------------
    %   ttblazer
    %---------------------------------------------------------------------
    \section{Application TTBlazer}
    \label{sec:ttblazer}
    \input parts/ttblazer.tex
    \pagebreak

    %---------------------------------------------------------------------
    %   conclusion
    %---------------------------------------------------------------------
    \section{Conclusion}
    \label{sec:concl}
    \input parts/conclusion.tex
    \pagebreak
    
    %---------------------------------------------------------------------
    %   appendices
    %---------------------------------------------------------------------
    \begin{appendices}
  		\section{File formats}
  		\label{app:formats}
  		\input parts/formats.tex
	\end{appendices}
	\pagebreak

%---------------------------------------------------------------------
%   BACKMATTER  ------------------------------------------------------
%---------------------------------------------------------------------
    %\backmatter

    %---------------------------------------------------------------------
    %   bibliography
    %---------------------------------------------------------------------

    \bibliographystyle{is-alpha}
    %compile latex, bibtex, latex, latex
    \bibliography{../bibl}{}
\end{document} 