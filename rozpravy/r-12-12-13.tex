%---------------------------------------------------------------------
%   documentclass
%---------------------------------------------------------------------
\documentclass[a4paper]{article}

%---------------------------------------------------------------------
%   packages
%---------------------------------------------------------------------
\usepackage{mathtools}
\usepackage[english]{babel}
\usepackage[enc=cp1250]{hrlatex}
\usepackage[T1]{fontenc} %pekne makcene
\usepackage{lmodern} %spolu s T1 smooth font!
\usepackage{hyperref} %odkazy
\usepackage{amsmath}
\usepackage{amsthm}
\usepackage{cite} %bibtex
\usepackage{enumitem}
\usepackage{titlesec} %section titles font size change
\usepackage{color} %for \definecolor
\usepackage{colortbl} %for \rowcolor command
\usepackage{eucal} %for nice letters like \mathcal{A}
\usepackage{tikz}
\usepackage{scalefnt}
\usepackage{booktabs}% http://ctan.org/pkg/booktabs
\usepackage{multirow}% http://ctan.org/pkg/multirow
\usepackage{pifont} %for ticks (check symbols)
\usepackage{listings}
\usepackage{floatflt} %to have tables and text beside
\usepackage{courier}

%---------------------------------------------------------------------
%   margins
%---------------------------------------------------------------------
\oddsidemargin 0.0in
\evensidemargin 0.0in
\textwidth 6.1in
\textheight 23.94cm
\topmargin -0.35in

%---------------------------------------------------------------------
%   various settings
%---------------------------------------------------------------------
\pgfrealjobname{r-12-12-13} % <-- NOTE: this needs to be the real document's basename
                        %     (else you'll only get an empty output file)

\newif\ifmine % introduce a switch for draft vs. final document
\minetrue

\newif\iffinal % introduce a switch for draft vs. final document
\finaltrue % use this to compile the final document
\iffinal
  \newcommand{\inputTikZ}[1]{%
    \input{#1.tikz}%
  }
\else
  \newcommand{\inputTikZ}[1]{%
    \beginpgfgraphicnamed{#1-external}%
    \input{#1.tikz}%
    \endpgfgraphicnamed%
  }
\fi

\setlist{nolistsep} %so that lists have normal spacing

\titleformat{\section}{\LARGE\bfseries}{\thesection}{1em}{} %section titles
\titleformat{\subsection}{\Large\bfseries}{\thesubsection}{1em}{} %subsection titles

\definecolor{tablehead}{RGB}{238,233,233} %nice smooth grey

\setlength{\parindent}{0pt} %we don't need no indentation

\graphicspath{{./pics/}} %picture dir

\definecolor{lstcolor}{RGB}{238,233,233}

\lstset{ %
    language=Octave,                % choose the language of the code
    basicstyle=\footnotesize\ttfamily,       % the size of the fonts that are used for the code
    numbers=left,                   % where to put the line-numbers
    numberstyle=\footnotesize,      % the size of the fonts that are used for the line-numbers
    stepnumber=1,                   % the step between two line-numbers. If it's 1 each line will be numbered
    numbersep=5pt,                  % how far the line-numbers are from the code
    backgroundcolor=\color{lstcolor},   % choose the background color. You must add \usepackage{color}
    showspaces=false,               % show spaces adding particular underscores
    showstringspaces=false,         % underline spaces within strings
    showtabs=false,                 % show tabs within strings adding particular underscores
    frame=single,	                % adds a frame around the code
    tabsize=2,	                    % sets default tabsize to 2 spaces
    captionpos=b,                   % sets the caption-position to bottom
    breaklines=true,                % sets automatic line breaking
    breakatwhitespace=false,        % sets if automatic breaks should only happen at whitespace
    title=\lstname,                 % show the filename of files included with \lstinputlisting; also try caption instead of title
    escapeinside={\%*}{*)},          % if you want to add a comment within your code
%    morekeywords={*,...}            % if you want to add more keywords to the set
	deletekeywords={all, null, length, path, function}
}

%---------------------------------------------------------------------
%   environments
%---------------------------------------------------------------------
\renewenvironment{abstract}[1]
{
	\Large
	\begin{center}
		\textbf{#1}
	\end{center}
	
	\normalsize
	
	\addtolength{\leftskip}{1in}
	\addtolength{\rightskip}{1in}
	\setlength{\parindent}{0in}
}
{
}

\newenvironment{itemizesp}
{
    \begin{itemize}
}
{
    \end{itemize}
}

\newcommand{\deftoken}{\boldmath{$\mathcal{DEFINITION}$}}
\newcommand{\restoken}{\boldmath{$\mathcal{RESULT}$}}
\newcommand{\dotoken}{\boldmath{$\mathcal{DO METHOD}$}}
\newcommand{\textbff}[1]{{\large \textbf{#1}}}
\newcommand{\tick}{\ding{52}}

\newtheorem{definition}{Definition}
\newtheorem{lemma}{Lemma}
\newtheorem{observation}{Observation}

%---------------------------------------------------------------------
%   magic code
%---------------------------------------------------------------------
% Here it is: the code that adjusts justification and spacing around caption.
\makeatletter
% http://www.texnik.de/floats/caption.phtml
% This does spacing around caption.
\setlength{\abovecaptionskip}{6pt}   % 0.5cm as an example
\setlength{\belowcaptionskip}{6pt}   % 0.5cm as an example
% This does justification (left) of caption.
\long\def\@makecaption#1#2{%
  \vskip\abovecaptionskip
  \sbox\@tempboxa{#1: #2}%
  \ifdim \wd\@tempboxa >\hsize
    #1: #2\par
  \else
    \global \@minipagefalse
    \hb@xt@\hsize{\box\@tempboxa\hfil}%
  \fi
  \vskip\belowcaptionskip}
\makeatother

%---------------------------------------------------------------------
%   document
%---------------------------------------------------------------------
\begin{document}
    \thispagestyle{empty}
    %---------------------------------------------------------------------
    %   topmatter
    %---------------------------------------------------------------------
    \title{\textbf{r-2012-12-13}}
    \author{Franti�ek Hajnovi� \\
    \texttt{ferohajnovic@gmail.com}}
    \date{\today}
    \maketitle

    \vskip 0.5cm

    %---------------------------------------------------------------------
    %   OUTLINE
    %---------------------------------------------------------------------
    \tableofcontents
    
    %---------------------------------------------------------------------
    %   Data analysis
    %---------------------------------------------------------------------    
    
	\section{Data analysis}
	Following is the first simple analysis of the data. Results were obtained by running following script in \textit{ttblazer} application: \\
	
	\begin{lstlisting} [caption={Script that obtained the results}]
#LOAD DATA
loadug ../procdata/cpsk/res/ug_cpza.ug
loadug ../procdata/cpsk/res/ug_cpru.ug
loadug ../procdata/zsr/res/ug_zsr.ug
loadug ../procdata/london/res/ug_london.ug
loadug ../procdata/svk/res/ug_svk.ug
loadtt -t ../procdata/cpsk/res/tt_cpza.tt
loadtt -t ../procdata/cpsk/res/tt_cpru.tt
loadtt -t ../procdata/zsr/res/tt_zsr.tt
#ANALYSE
an ugconn ug 0 -partdet
an ugdegs ug 0 -degdet
an ugconn ug 1 -partdet
an ugdegs ug 1 -degdet
an ugconn ug 2 -partdet
an ugdegs ug 2 -degdet
an ugconn ug 3 -partdet
an ugdegs ug 3 -degdet
an ugconn ug 4 -partdet
an ugdegs ug 4 -degdet
showug 0
showug 1
showug 2
showug 3
showug 4
showtt 0
showtt 1
showtt 2
quit
	\end{lstlisting}
	
	\begin{table}[h]{
		\scriptsize
	    \begin{tabular}{l|c|c|c|c}
		%legend
		\hline
			\rowcolor{tablehead}
           	\textbf{Name} & 
           	\textbf{Type} & 
           	\textbf{\# nodes} & 
           	\textbf{\# arcs} & 
           	\textbf{Load time} \\
        %data
        \hline
            ug\_cpza.ug & Regional bus & 1128 & 2034 & 0.4s \\
            ug\_cpru.ug & Regional bus & 877 & 1784 & 0.4s \\
            ug\_zsr.ug & Country-wide rails& 233 & 588 & 0.1s \\
            ug\_london.ug & Underground rails & 321 & 732 & 0.1s \\
            ug\_svk.ug & Road network & 181386 & 425829 & 7.754s \\
        \end{tabular}}
		\caption{\label{tab:ugmain} Underlying graphs - main properties}
        \normalsize
	\end{table}

	\begin{table}[h]{
		\scriptsize
	    \begin{tabular}{l|c|c|c|c|c|c|c}
		%legend
		\hline
			\rowcolor{tablehead}
           	\textbf{Name} & 
           	\textbf{\# nodes} & 
           	\textbf{\# arcs} & 
           	\textbf{Avg deg.} & 
           	\textbf{Min deg.} & 
           	\textbf{Max deg.} & 
           	\textbf{Degrees} &
           	\textbf{Analysis time} \\      	
        %data
        \hline
            ug\_cpza.ug & 1128 & 2034 & 1.80319 & 0 & 24 & 
\parbox{4cm}{
0: 78x,
1: 536x,
2: 297x,
3: 115x,
4: 48x,
5: 25x,
6: 10x,
7: 7x,
8: 5x,
9: 2x,
10: 1x,
12: 1x,
14: 1x,
15: 1x,
24: 1x
} & 0.0s \\ \hline
            ug\_cpru.ug & 877 & 1784 & 2.03421 & 0 & 17 & 
\parbox{4cm}{
0: 54x,
1: 405x,
2: 202x,
3: 86x,
4: 55x,
5: 34x,
6: 17x,
7: 11x,
8: 3x,
9: 2x,
10: 1x,
11: 1x,
12: 1x,
13: 2x,
14: 1x,
16: 1x,
17: 1x
} & 0.0s \\ \hline
            ug\_zsr.ug & 233 & 588 & 2.52361 & 0 & 12 & 
\parbox{4cm}{
1: 42x,
2: 119x,
3: 28x,
4: 14x,
5: 14x,
6: 6x,
7: 3x,
8: 1x,
9: 1x,
12: 2x
} & 0.0s \\ \hline
            ug\_london.ug & 321 & 732 & 2.28037 & 0 & 7 & 
\parbox{4cm}{
0: 3x,
1: 25x,
2: 230x,
3: 25x,
4: 27x,
5: 5x,
6: 3x,
7: 3x
} & 0.0s \\ \hline
            ug\_svk.ug & 181386 & 425829 & 2.34764 & 0 & 6 & 
\parbox{4cm}{
0: 96x,
1: 46469x,
2: 34998x,
3: 90084x,
4: 9586x,
5: 150x,
6: 3x
} & 0.25s \\ 
        \end{tabular}}
		\caption{\label{tab:ugdeg} Underlying graphs - degrees}
        \normalsize
	\end{table}
	
	\begin{table}[h!]{
		\scriptsize
	    \begin{tabular}{l|c|c|c|c|c|c}
		%legend
		\hline
			\rowcolor{tablehead}
           	\textbf{Name} & 
           	\textbf{\# nodes} & 
           	\textbf{\# arcs} & 
           	\textbf{\# connected} &
           	\textbf{\# of conn. comps} &
           	\textbf{\# of component sizes} &
           	\textbf{Analysis time} \\
        %data
        \hline
            ug\_cpza.ug & 1128 & 2034 & false & 21 & 
\parbox{4cm}{
1108, 1, 1, 1, 1, 1, 1, 1, 1, 1, 1, 1, 1, 1, 1, 1, 1, 1, 1, 1, 1
} & 0.2s \\ \hline
            ug\_cpru.ug & 877 & 1784 & false & 7 &
\parbox{4cm}{
871, 1, 1, 1, 1, 1, 1
} & 0.1s \\ \hline
            ug\_zsr.ug & 233 & 588 & true & 1 & 
\parbox{4cm}{
233
} & 0.0s \\ \hline
            ug\_london.ug & 321 & 732 & false & 4 & 
\parbox{4cm}{
318, 1, 1, 1
} & 0.1s \\ \hline
            ug\_svk.ug & 181386 & 425829 & false & 871 & 
\parbox{4cm}{
178187, 134, 105, 30, 29, ...
} & 4.478s \\
        \end{tabular}}
		\caption{\label{tab:ugconn} Underlying graphs - connectivity}
        \normalsize
	\end{table}
	
	\begin{table}[h!]{
		\scriptsize
	    \begin{tabular}{l|c|c|c}
		%legend
		\hline
			\rowcolor{tablehead}
           	\textbf{Name} & 
           	\textbf{Type} &
           	\textbf{\# el. conn.} &
           	\textbf{Load time} \\
        %data
        \hline
            tt\_cpza & Regional bus & 61747 & 4.451s \\
            tt\_cpru & Regional bus & 38540 & 2.275s \\
            tt\_zsr & Country-wide rails & 932052 & 66.6668s \\
        \end{tabular}}
		\caption{\label{tab:ttmain} Timetables - main properties}
        \normalsize
	\end{table}
	
	\pagebreak
	
	%---------------------------------------------------------------------
    %   OPEN POINTS
    %---------------------------------------------------------------------    
    \section{Open points}
    \begin{itemize}
    		\item Hierarchy of express lines $\rightarrow$ what properties can be propagated in time-expansion?
    		\item Instant cost function - more formal and details
    \end{itemize}
    
    %---------------------------------------------------------------------
    %   TO DO
    %---------------------------------------------------------------------    
    \section{To do}
    
    \begin{itemize}
    		\item Road network of SVK - process data \tick
    		\item United airlines extract data
    		\item Continue the diagnostic program
    		\item Properties propagation in simple timetables
    		\item Machine learning
    \end{itemize}

    %---------------------------------------------------------------------
    %   bibliography
    %---------------------------------------------------------------------
    \pagebreak
    \bibliographystyle{is-alpha}
    %compile latex, bibtex, latex, latex
    \bibliography{../bibl}{}
\end{document}
