%---------------------------------------------------------------------
%   documentclass
%---------------------------------------------------------------------
\documentclass[a4paper]{article}

%---------------------------------------------------------------------
%   packages
%---------------------------------------------------------------------
\usepackage{mathtools}
\usepackage[english]{babel}
\usepackage[enc=cp1250]{hrlatex}
\usepackage[T1]{fontenc} %pekne makcene
\usepackage{lmodern} %spolu s T1 smooth font!
\usepackage{hyperref} %odkazy
\usepackage{amsmath}
\usepackage{amsthm}
\usepackage{cite} %bibtex
\usepackage{enumitem}
\usepackage{titlesec} %section titles font size change
\usepackage{color} %for \definecolor
\usepackage{colortbl} %for \rowcolor command
\usepackage{eucal} %for nice letters like \mathcal{A}
\usepackage{tikz}
\usepackage{scalefnt}
\usepackage{booktabs}% http://ctan.org/pkg/booktabs
\usepackage{multirow}% http://ctan.org/pkg/multirow
\usepackage{pifont} %for ticks (check symbols)
\usepackage{listings}
\usepackage{floatflt} %to have tables and text beside
\usepackage{courier}

%---------------------------------------------------------------------
%   margins
%---------------------------------------------------------------------
\oddsidemargin 0.0in
\evensidemargin 0.0in
\textwidth 6.1in
\textheight 23.94cm
\topmargin -0.35in

%---------------------------------------------------------------------
%   various settings
%---------------------------------------------------------------------
\pgfrealjobname{r-12-11-08} % <-- NOTE: this needs to be the real document's basename
                        %     (else you'll only get an empty output file)

\newif\ifmine % introduce a switch for draft vs. final document
\minetrue

\newif\iffinal % introduce a switch for draft vs. final document
\finaltrue % use this to compile the final document
\iffinal
  \newcommand{\inputTikZ}[1]{%
    \input{#1.tikz}%
  }
\else
  \newcommand{\inputTikZ}[1]{%
    \beginpgfgraphicnamed{#1-external}%
    \input{#1.tikz}%
    \endpgfgraphicnamed%
  }
\fi

\setlist{nolistsep} %so that lists have normal spacing

\titleformat{\section}{\LARGE\bfseries}{\thesection}{1em}{} %section titles
\titleformat{\subsection}{\Large\bfseries}{\thesubsection}{1em}{} %subsection titles

\definecolor{tablehead}{RGB}{238,233,233} %nice smooth grey

\setlength{\parindent}{0pt} %we don't need no indentation

\graphicspath{{./pics/}} %picture dir

\definecolor{lstcolor}{RGB}{238,233,233}

\lstset{ %
    language=Octave,                % choose the language of the code
    basicstyle=\footnotesize\ttfamily,       % the size of the fonts that are used for the code
    numbers=left,                   % where to put the line-numbers
    numberstyle=\footnotesize,      % the size of the fonts that are used for the line-numbers
    stepnumber=1,                   % the step between two line-numbers. If it's 1 each line will be numbered
    numbersep=5pt,                  % how far the line-numbers are from the code
    backgroundcolor=\color{lstcolor},   % choose the background color. You must add \usepackage{color}
    showspaces=false,               % show spaces adding particular underscores
    showstringspaces=false,         % underline spaces within strings
    showtabs=false,                 % show tabs within strings adding particular underscores
    frame=single,	                % adds a frame around the code
    tabsize=2,	                    % sets default tabsize to 2 spaces
    captionpos=b,                   % sets the caption-position to bottom
    breaklines=true,                % sets automatic line breaking
    breakatwhitespace=false,        % sets if automatic breaks should only happen at whitespace
    title=\lstname,                 % show the filename of files included with \lstinputlisting; also try caption instead of title
    escapeinside={\%*}{*)}          % if you want to add a comment within your code
%    morekeywords={*,...}            % if you want to add more keywords to the set
}

%---------------------------------------------------------------------
%   environments
%---------------------------------------------------------------------
\renewenvironment{abstract}[1]
{
	\Large
	\begin{center}
		\textbf{#1}
	\end{center}
	
	\normalsize
	
	\addtolength{\leftskip}{1in}
	\addtolength{\rightskip}{1in}
	\setlength{\parindent}{0in}
}
{
}

\newenvironment{itemizesp}
{
    \begin{itemize}
}
{
    \end{itemize}
}

\newcommand{\deftoken}{\boldmath{$\mathcal{DEFINITION}$}}
\newcommand{\restoken}{\boldmath{$\mathcal{RESULT}$}}
\newcommand{\dotoken}{\boldmath{$\mathcal{DO METHOD}$}}
\newcommand{\textbff}[1]{{\large \textbf{#1}}}
\newcommand{\tick}{\ding{52}}

\newtheorem{definition}{Definition}
\newtheorem{lemma}{Lemma}
\newtheorem{observation}{Observation}

%---------------------------------------------------------------------
%   magic code
%---------------------------------------------------------------------
% Here it is: the code that adjusts justification and spacing around caption.
\makeatletter
% http://www.texnik.de/floats/caption.phtml
% This does spacing around caption.
\setlength{\abovecaptionskip}{6pt}   % 0.5cm as an example
\setlength{\belowcaptionskip}{6pt}   % 0.5cm as an example
% This does justification (left) of caption.
\long\def\@makecaption#1#2{%
  \vskip\abovecaptionskip
  \sbox\@tempboxa{#1: #2}%
  \ifdim \wd\@tempboxa >\hsize
    #1: #2\par
  \else
    \global \@minipagefalse
    \hb@xt@\hsize{\box\@tempboxa\hfil}%
  \fi
  \vskip\belowcaptionskip}
\makeatother

%---------------------------------------------------------------------
%   document
%---------------------------------------------------------------------
\begin{document}
    \thispagestyle{empty}
    %---------------------------------------------------------------------
    %   topmatter
    %---------------------------------------------------------------------
    \title{\textbf{r-2012-11-08}}
    \author{Franti�ek Hajnovi� \\
    \texttt{ferohajnovic@gmail.com}}
    \date{\today}
    \maketitle

    \vskip 0.5cm

    %---------------------------------------------------------------------
    %   OUTLINE
    %---------------------------------------------------------------------
    \tableofcontents

    %---------------------------------------------------------------------
    %   processed data
    %---------------------------------------------------------------------
    \section{Processed real data}
    Table ~\ref{tab:ttfiles} summarizes the so far processed timetables. The regional bus timetables provided by \textit{cp.sk} are the examples of more local timetables with underlying network being the road network. The time range was so far set to 1 day, assuming that each bus operates the same way every day. This is not a correct assumption (some buses may operate e.g. only during weekends), but this way we save space and the resulting timetable should not differ much in properties from the real one. We may later however try to extend the time range to one week and see the changes in properties. \\
    
    Timetable from \textit{zsr.sk} covers a full year, as connections had a reference to a calendar, stating if the train operates on a given date. That is also the reason for the large number of elementary connections. \\
    
	Timetable of \textit{United airlines} requires more processing, but should be later added as it provides yet another type of timetable. \\    
    
	\begin{table}[h]{
		\scriptsize
	    \begin{tabular}{l|c|c|c|c|c|c}
		%legend
		\hline
			\rowcolor{tablehead}
           	\textbf{Name} & \textbf{Type of network} & \textbf{Provided by} & \textbf{\# of stations} & \parbox{2.5cm}{\begin{center} \textbf{\# of elementary connections} \end{center}} & \textbf{Time range} & \textbf{Note} \\
        %data
        \hline
            tt\_cpza & Regional bus & cp.sk & 1128 & 68096 & 1 day & Area of �ilina\\
            tt\_cpru & Regional bus & cp.sk & 877 & 43069 & 1 day & Area of Ru�omberok\\
            tt\_zsr & Country-wide rails & zsr.sk & 233 & 932052 & 1 year &  \\
        \end{tabular}}
		\caption{\label{tab:ttfiles}Timetable files}
        \normalsize
	\end{table}
	
	Timetable files have a simple form: \\
	
	\begin{lstlisting}[caption={Timetable files form}]
		7                      //number of elementary connections
		A B 0 10:00 0 10:45    //FROM TO DAY(depart) TIME(depart) DAY(arrive) TIME(arrive)
		A B 0 11:00 0 11:45
		A B 0 12:00 0 12:45
		A C 0 9:30 0 10:00
		A C 0 10:15 0 10:45
		C D 0 11:00 0 11:30
		C D 0 13:00 0 13:30
	\end{lstlisting}
	
	Table \ref{tab:ugfiles} summarizes the so far processed maps (or underlying graphs) of networks, on top of which we can later build random timetables with given properties. The maps were created by extracting stations from the timetable and making an edge between the pair where a connection is operating, thus getting the underlying structure of the timetable. Note that we will not really get the real map of the (e.g.) rails - with our approach, there will be a direct link between a pair of large (but distant) cities even though many smaller stations exist on the way. This is due to express lines that do not stop at these smaller stations. If the real underlying map should be desirable, we can later remove these ``express links'', although we cannot be sure that the express line did not really follow a new, faster road/railway without any smaller stops on the way (that should not happen in case of railways). \\
	
	Map of the \textit{road network of Slovakia} should be added, as well as the underlying map of \textit{United airlines}. Also, currently the edges of the underlying graphs have null lengths - the travel time of the fastest connection on the edge can be used instead. \\ 
	
	\begin{table}[h]{
		\scriptsize
	    \begin{tabular}{l|c|c|c|c|c|c}
		%legend
		\hline
			\rowcolor{tablehead}
           	\textbf{Name} & \textbf{Type of network} & \textbf{Provided by} & \textbf{\# of stations} & \textbf{\# of connections} & \textbf{Notion of lines} & \textbf{Note} \\
        %data
        \hline
            ug\_cpza & Regional bus & cp.sk & 1128 & 2034 & \tick & Area of �ilina\\
            ug\_cpru & Regional bus & cp.sk & 877 & 1784 & \tick & Area of Ru�omberok\\
            ug\_zsr & Country-wide rails & zsr.sk & 233 & 588 & \tick &  \\
            ug\_london & Underground rails & queensu.ca & 321 & 732 & \tick &  \\
        \end{tabular}}
		\caption{\label{tab:ugfiles} Underlying graph files}
        \normalsize
	\end{table}
	
	Underlying graph files have the following form: \\
	
	\begin{lstlisting}[caption={Underlying graph files form}]
		4                                      //number of stations
		5                                      //number of edges
		A [45 32]                              //name of the station, optional coordinates
		B
		C [56 34]
		D
		A B 57 Northern                        //FROM TO edge length, list of lines operating on the edge
		A C null Picadilly Victoria            //edge length may be null (will be e.g. random, or calculated from coordinates)
		C B 45 Circle Jubilee Picadilly
		C D 32 null                            //list of lines may be also null
		D A 90 Metropolitan Victoria

	\end{lstlisting}

    %---------------------------------------------------------------------
    %   formal overtaking/express lines
    %---------------------------------------------------------------------
    \section{Overtaking and express lines}
	
	In this section, we will define and discuss two notions: \textit{overtaking} and \textit{express lines}. For clarity, we repeat the definition of the timetable:
	
	\begin{definition}
        \textbf{Timetable on a graph $G$} \\
        A timetable on a given graph $G$ is a set $T_{G} = \{(x, y, p, q)|(x, y) \in E_{G}, \; p, q \in N, \; p < q \}$. 
		\begin{itemize}
			\item An element of T is called an \textbf{elementary connection} and $x$ [$y$] and $p$ [$q$] are the \textbf{departure [arrival] node} and \textbf{time}. 
        		\item Graph $G$ is the \textbf{underlying graph} of timetable $T$.
		\end{itemize}		        
    \end{definition}
    
    \begin{definition}
        \textbf{Connection from $a$ to $b$ in a given timetable $T_{G}$} \\
        A connection from $a$ to $b$ in a given timetable $T_{G}$ is a sequence of \emph{elementary connections} $(e_{1}, e_{2}, ..., e_{k}), \; k \geq 1, \; e_{i} = (e_{x}^{i}, e_{y}^{i}, e_{p}^{i}, e_{q}^{i})$, such that $e_{x}^{1} = a$, $e_{y}^{k} = b$ and $\forall i \in \{2, ..., k\}: (e_{x}^{i} = e_{y}^{i - 1}, \; e_{p}^{i} \geq e_{q}^{i - 1})$
        \begin{itemize}
        		\item Connection \textbf{starts} at the \emph{departure time} $e_{p}^{1}$.
        		\item \textbf{Length} of the connection is $e_{q}^{k} - e_{p}^{1}$.        
        \end{itemize}
    \end{definition}
    
    \begin{table}{
        \small
        \begin{tabular}{c|c|c|c}
            %legend
            \hline
                \rowcolor{tablehead}
%                place & h & nh & \\
                \multicolumn{2}{>{\columncolor{tablehead}}c|}{\textbf{Place}} & \multicolumn{2}{>{\columncolor{tablehead}}c}{\textbf{Time}} \\
                \hline
                \rowcolor{tablehead}
                \textbf{From} & \textbf{To} & \textbf{Departure} & \textbf{Arrival} \\
            %data
            \hline
				A & B & 11:00 & 11:45 \\
				A & C & 9:30 & 10:00 \\
				A & C & 10:15 & 10:45 \\
				A & C & 10:30 & 11:00 \\
				A & C & 10:40 & 11:05 \\
				C & D & 11:00 & 11:40 \\
				C & D & 11:05 & 11:25 \\
				C & B & 11:20 & 12:00 \\
				D & A & 10:00 & 11:00 \\
        \end{tabular}}
        \caption{\label{tab:tt} An example of a timetable}
        \normalsize
    \end{table}
    
    \begin{figure}[h!]
        \begin{center}
            \inputTikZ{./tikzpics/basegraph}
        \end{center}
        \caption{\label{fig:bg} Underlying graph of the timetable in ~\ref{tab:tt}. Note, that there does not have to be a connection on every arc of the underlying graph (e.g. there is an arc from $D$ to $C$, but no connection operates here)}
    \end{figure}    
    
    \begin{figure}[h!]
        \begin{center}
            \inputTikZ{./tikzpics/ttgrexp_clear}
        \end{center}
        \caption{\label{fig:ttgrexp_clear} Graph representing the timetable in ~\ref{tab:tt}}
    \end{figure}
    
    Overtaking, simply stated, means that there are connections in the timetable that overtake one another - that is by choosing a later connection, we will actually arrive sooner to the destination. Note, a connections overtake one another only if they follow the same route (the sequence of traversed arcs is the same) - otherwise we are talking simply about choosing a quicker connection, which is actually the task the search engines perform. See picture \ref{fig:ttgrexp} for better explanation.
    
    \begin{definition}
        \textbf{Overtaking} \\
        A timetable $T_{G}$ has overtaking property $\iff$ there exist two connections $e$ and $f$ from $a$ to $b$ in $T_{G}$ such that $e = (e_{1}, e_{2}, ..., e_{k})$, $f = (f_{1}, f_{2}, ..., f_{k})$, $\forall i$ $e_{x}^{i} = f_{x}^{i}$, $e_{y}^{i} = f_{y}^{i}$ and $e_{p}^{1} < f_{p}^{1}$ but $e_{q}^{k} > f_{q}^{k}$.
    \end{definition}
    
    This definition can be however simplified if we realize, that if a connection $e$ overtakes connection $f$, it must do so on a single specific arc. Thus a simpler definition (but equivalent one) would consider only elementary connections:
    
    \begin{definition}
        \textbf{Overtaking (simpler form)} \\
        A timetable $T_{G}$ has overtaking property $\iff$ there exist two elementary connections $e$ and $f$ from $a$ to $b$ in $T_{G}$ such that $e_{p} < f_{p}$ but $e_{q} > f_{q}$.
    \end{definition}
    
    What does the timetable having overtaking property influence? For oriented weighted graphs (representing e.g. static road network), many fast and precise algorithms were developed for finding shortest paths. These algorithms often use a modified Dijkstra's algorithm at query time, enhanced by additional information obtained during preprocessing phase. As far as the used Dijkstra's algorithm was unidirectional, it can be straightforwardly adapted to a time-dependent scenario (although there will most probably be various changes in the preprocessing phase as well) - each time the time-dependent Dijkstra considers an arc, it looks at the arc's cost function to obtain the traversal cost at the given moment. In case there is no overtaking in the timetable, getting the traversal cost of the given arc simply means looking into the timetable and finding the first elementary connection that traverses the arc, which requires no preprocessing. Otherwise we would need to pre-process the graph to obtain cost functions for each arc. Such preprocessing could be done in time linear of the number of elementary connections.\\
    
    We now move on to the other notion - express lines. Informally, express lines are connections that bypass several stops on the way. In our definition of timetable graph, we do not allow such connections so we extend the definition.
    
    \begin{definition}
        \textbf{Timetable with express lines on a graph $G$} \\
        A timetable with express lines on a given graph $G$ is a set $T_{G} = \{(x, y, p, q)| \text{there is a path from } x \text{ to } y \text{ in } G, \; p, q \in N, \; p < q \}$.
        \begin{itemize}
	        \item Tuples $(x, y, p, q)$ where $(x, y) \in E_{G}$ are \textbf{elementary connections}, the remaining are \textbf{express connections}.
    	    		\item Such $(x, y)$ that $(x, y) \not \in E_{G}$ but there is a path from $x \text{ to } y \text{ in } G$ will be called \textbf{express lines}.        
        \end{itemize}
    \end{definition}
    
    \begin{table}{
        \small
        \begin{tabular}{c|c|c|c}
            %legend
            \hline
                \rowcolor{tablehead}
%                place & h & nh & \\
                \multicolumn{2}{>{\columncolor{tablehead}}c|}{\textbf{Place}} & \multicolumn{2}{>{\columncolor{tablehead}}c}{\textbf{Time}} \\
                \hline
                \rowcolor{tablehead}
                \textbf{From} & \textbf{To} & \textbf{Departure} & \textbf{Arrival} \\
            %data
            \hline
				A & D & 10:00 & 11:00 \\
        \end{tabular}}
        \caption{\label{tab:ttexpress} An added express connection to our timetable in \ref{tab:tt}}
        \normalsize
    \end{table}
    
    \begin{figure}[h!]
        \begin{center}
            \inputTikZ{./tikzpics/basegraph_express}
        \end{center}
        \caption{\label{fig:bg} Underlying graph of the timetable in ~\ref{tab:tt} with added express line from $A$ to $D$}
    \end{figure}    
    
    \begin{figure}[h!]
        \begin{center}
            \inputTikZ{./tikzpics/ttgrexp}
        \end{center}
        \caption{\label{fig:ttgrexp} Graph representing the timetable in ~\ref{tab:tt} with added express connection from $A$ to $D$ (marked with \textcolor{red}{red} color). An example of overtaking is marked with \textcolor{green}{green} color. With \textcolor{orange}{orange} color we have marked two connections that \textit{do not} represent overtaking - waiting at node $A$ for the connection at $11:00$ is simply a smarter choice.}
    \end{figure}
    
    Now we can formulate the following observation:
    
    \begin{observation}
        \textbf{} \\
        Let $G$ be a graph with the vertex set $V$, $T_{G}$ timetable on $G$ and $H$ a connected graph with the vertex set $V$. There exists a timetable with express lines $T_{H}$ such that $T_{H} = T_{G}$.
    \end{observation}
    
   	\begin{proof}
   		The proof is trivial: By the definition of timetable with express lines, there may be a connection between any pair of vertices in the underlying graph $H$, as it is connected. Thus for every elementary connection $(x, y, p, q) \in T_{G}$ we will simply have the same (\textit{either} elementary \textit{or} express) connection $(x, y, p, q)$ in $T_{H}$
   	\end{proof}
   	
   	Informally, this observation just states, that with express lines we can forget about underlying structure of the timetable and consider it as a complete graph. That would however imply that no property of the underlying graph $G$ can be generally propagated to the graph representing the timetable on $G$ (not even to certain extent). Unfortunately, express lines are not uncommon in real-world scenarios. On the other hand, usage of express lines is limited in practice and generally express lines act as a shortcut layer on top of the real underlying network. One can imagine this as a hierarchy starting from the road network with nodes being the intersections. Next layer would be the local bus lines, then express inter-city lines etc... See picture \ref{fig:bghier} for demonstration.
   	
   	\begin{figure}[h!]
        \begin{center}
            \inputTikZ{./tikzpics/bghier}
        \end{center}
        \caption{\label{fig:bghier} An example of the ``shortcut'' layers. Original underlying graph representing structure of the map has black edges. However bigger towns are also connected by express lines depicted by \textcolor{green}{green} color. The two biggest cities are connected by yet another express line, marked \textcolor{red}{red}.}
    \end{figure}
    
    \pagebreak
    
    %---------------------------------------------------------------------
    %   adjustment of Gavoille
    %---------------------------------------------------------------------    
    \section{Adjustment of Gavoille's algorithm for graphs with $r(n)$ separator ~\cite{distlabel04} for time-dependent scenario}
	In ~\cite{distlabel04}, an algorithm answering distance queries in oriented weighted graphs is presented. The algorithm pre-processes the input graph, producing the so-called distance labels for each vertex. After preprocessing, a distance query between a pair of vertices $u$ and $v$ can be answered in logarithmic (TODO, correct this) time using only the distance labels of $u$ and $v$. \\
	
	The algorithm takes advantage of recursive separators in the graph. The \textit{size of the separator} and \textit{how quickly we can find it} are two factors that influence the efficiency of the algorithm:
	\begin{itemize}
		\item Size of the separator influences the resulting size of the distance labels. If we define
		\[
			\mathbf{R(n)} = \sum_{i = 0}^{\log_{3/2} n} r(n(2/3)^i)
		\]
		where $r(n)$ is the size of the recursive separator, the resulting total size of the preprocessed distance labels is
		\[
			\mathcal{O}(n R(n) \log W + n \log^{2} n)
		\]
		with $W$ being the maximum arc length in the graph.
		\item The time it takes us to find the separator, on the other hand, influences the preprocessing time. E.g., in planar graphs, we can find a recursive separator of size $\mathcal{O} (\sqrt n)$ (\cite{separators2} in linear time ($\mathcal{O} (n)$). That leads to the preprocessing time of $\mathcal{O} (n^{3/2} \log n)$.
	\end{itemize}
	\hspace*{\fill}	
	
	We will now briefly describe how the Gavoille's algorithm works (more details can be found in ~\cite{distlabel04}). 
	\begin{enumerate}
		\item A separator $S$ is found and the graph is split into corresponding components
		\item Each node in the component gets the list of distances to all nodes from $S$
		\item Each node in $S$ also gets the list of distances to all other nodes from $S$
		\item A node in a component gets the identifier of the component
		\item We proceed recursively in components
	\end{enumerate}
 	\hspace*{\fill}
 
	After the preprocessing, answering a distance query between a pair of vertices $u$ and $v$ is simple. If $u$ and $v$ are from a different top level component, the shortest path connecting them must pass through the top level separator. We thus consider all of its vertices and find the distance as
	\[
		d(u, v) = min_{s \in S} \{d(u, s) + d(s, v)\}
	\]
	In case $u$ and $v$ are from the same top level component $C$, their corresponding shortest path may still pass through the top level separator, but it may also be completely within the component $C$. Thus we take the minimum out of these two values. For more details, please refer to ~\cite{distlabel04}, section \textit{2.2}. \\
	
	We will now be interested, if this algorithm can be adjusted to work in timetable scenarios, that is, on graphs representing timetables.\\
	
	*TODO*
    
    %---------------------------------------------------------------------
    %   OPEN POINTS
    %---------------------------------------------------------------------    
    \section{Open points}
    \begin{itemize}
    		\item What does overtaking really affect?
    		\item Hierarchy of express lines $\rightarrow$ what properties can be propagated in time-expansion?
    		\item Find use of machine learning?
    		\begin{itemize}
    			\item Regression - 
    			\item Separation - 
    			\item Clustering - e.g. for multi-level partition in SHARC
    			\item Decision trees - 
    			\item Neural network - e.g. for property prediction
    			\item Online algorithms - 
    		\end{itemize}
    		\item General formula for preprocessing time in Gavoille's algorithm
    \end{itemize}
    
    %---------------------------------------------------------------------
    %   TO DO
    %---------------------------------------------------------------------    
    \section{To do}
    
    \begin{itemize}
    		\item Extract road network of SVK \tick
    		\item Replace null lengths in underlying graphs of existing timetables \tick
    		\item Adjustment of Gavoille's algorithm for planar graphs for time-dependent scenario
    		\item United airlines extract data
    		\item Road network of SVK - process data
    		\item Start work on the diagnostic program
    \end{itemize}

    %---------------------------------------------------------------------
    %   bibliography
    %---------------------------------------------------------------------
    \pagebreak
    \bibliographystyle{is-alpha}
    %compile latex, bibtex, latex, latex
    \bibliography{../bibl}{}
\end{document}
